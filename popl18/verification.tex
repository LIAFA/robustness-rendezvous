%!TEX root = main.tex
\section{Checking Synchronizability}

%Verifying synchronizability for a message passing system using a monitor that maintains the conflict-graph of a trace 
%is difficult because it requires dealing with executions where the message buffers are unbounded. 
%For instance, even if the processes are finite-state, this would not lead to a decision procedure. 
We show in this section that checking synchronizability can be reduced 
to a reachability problem in a system that executes under the \emph{synchronous} semantics, 
where the message buffers cannot grow beyond some fixed bound. The main idea is to show that every
\emph{borderline} synchronizability violation (for which every strict prefix is synchronizable) of a system $\mathcal{S}$ can be ``simulated''~\footnote{We refer to the standard notion of (stuttering) simulation where (sequences of) transitions in $\mathcal{S}$ are mapped to transitions of $\mathcal{S'}$.}
by the synchronous semantics of a system $\mathcal{S'}$ where the reception of exactly one message is delayed (w.r.t. the synchronous semantics).
Then, we give a monitor which observes executions of $\mathcal{S'}$ and identifies synchronizability violations
(it goes to an error state whenever it finds such an execution).

%This result is based on the following ideas:
%\begin{itemize}
%	\item since the set of synchronizable executions is prefix-closed (Lemma~\ref{lem:pref_closed}), it is enough to look for \emph{borderline} violations, i.e., executions which
%are not synchronizable but for which every strict prefix is synchronizable,
%	\item starting from the original system $\mathcal{S}$, define a new system $\mathcal{S'}$ whose synchronous semantics ``simulates'' all the borderline violations of $\mathcal{S}$. 
%	\item define a monitor $\mathcal{M}_{\mathit{causal}}$, which identifies executions of $\mathcal{S'}$ under the synchronous semantics
%which are not executions of the original system $\mathcal{S}$ ($\mathcal{M}_{\mathit{causal}}$ goes to an error state whenever it finds such 
%an execution), 
%	\item define a \emph{monitor} $\mathcal{M}_{\mathit{viol}}$, which identifies synchronizability violations of $\mathcal{S}$ 
%($\mathcal{M}_{\mathit{viol}}$ goes to an error state whenever it finds such an execution),
%	\item establish that $\mathcal{S}$ is $k$-synchronizable whenever the synchronous semantics of $\mathcal{S'}\paral \mathcal{M}_{\mathit{causal}}\paral \mathcal{M}_{\mathit{viol}}$
%doesn't reach a configuration where $\mathcal{M}_{\mathit{viol}}$ is in an error state, but $\mathcal{M}_{\mathit{causal}}$ is \emph{not} in an error state 
%(i.e., reaching such a configuration would mean that the synchronous semantics of $\mathcal{S'}$ admits an execution which is valid according to $\mathcal{S}$ but not synchronizable). 
%\end{itemize}

%\subsection{Synchronizability Violation Patterns}
%
%
%\begin{figure}[t]
%Causal delivery violation:
%\begin{align*}
%\send{1}{p_1,q,\_}\leadsto \send{2}{p_2,q,\_}\mbox{ and }
%\rec{2}{q,\_}  \leadsto_{lossy}  \rec{1}{q,\_} 
%\end{align*}
%Exchange pattern:
%\begin{align*}
%\send{1}{p_1,q_1,\_}\leadsto_{lossy} \rec{2}{q_2,\_}\mbox{ and }
%\send{2}{p_2,q_2,\_}\leadsto_{lossy} \rec{1}{q_1,\_}
%\end{align*}
%\caption{Synchronizability violation patterns.}
%\label{fig:patterns}
%\end{figure}
%
%We show that any trace violating synchronizability contains one of the two violation patterns in Figure~\ref{fig:patterns}.
%Intuitively, the first pattern describes a violation to \emph{causal delivery} where two causally related messages sent to the same process
%are received in an order different from that in which they were sent. Here, the causal order between messages is defined
%by the paths in the action graph (ignoring unmatched sends, the predicate $\leadsto_{lossy}$ is exactly $\leadsto$). 
%The second violation pattern describes a situation in which two messages are sent concurrently 
%and each message is received after the other one is sent. This situation arises for instance in distributed protocols which alternate
%between phases in which all processes send messages to their peers and phases in which they receive messages from their peers. 
%In these cases, the sends and receives in the exchange pattern are causally related since they are executed by the same process.
%
%We say that a trace contains a causal delivery violation or an exchange pattern if it contains the actions and the action-graph paths specified in Figure~\ref{fig:patterns}.
%In the context of matched traces, the predicate $\leadsto_{lossy}$ is exactly $\leadsto$ (describing action-graph paths). Otherwise, the interpretation of $\leadsto_{lossy}$ takes into consideration the fact that the receive action in the right-hand side can occur in the completion of an unmatched trace. 
%Thus, $a \leadsto_{lossy}  \rec{1}{q,\_}$ holds in a trace $t$ when $a$ is an action of $t$ and either $a\leadsto \rec{1}{q,\_}$ (which implies that $\rec{1}{q,\_}$ is also an action of $t$), or $t$ contains an action $b$ such that $\<proc>(b)=q$ and  $a \leadsto b$, or $\<proc>(a)=q$. 
%
%\begin{theorem}\label{th:patt}
%A trace is not synchronizable if{f} it contains a causal delivery violation or an exchange pattern.
%\end{theorem}
%\begin{proof}
%The ``only if'' direction is trivial. For both violation patterns, the two pairs of matching send/receive actions (or one pair and an unmatched send, in the case of the causal delivery violations) will define a conflict graph cycle.
%
%For the ``if'' direction, let $t$ be a non-synchronizable trace. Let us first assume that $t$ is matched. Then, the conflict graph of $t$ contains two nodes $v_1$ and $v_2$ such that there is a path from $v_1$ to $v_2$ and vice-versa. 
%%By an abuse of notation, assume that the action graph contains also edges 
%%$v_1\leadsto v_2$ and $v_2\leadsto v_1$.
%%Let $G_t'$ be the action graph of $t$ extended with edges between every two nodes $u$ and $v$ such that 
%%$\mathrm{act}(u)\in R_{id}$, $\mathrm{act}(v)\in S_{id}$ is an unmatched send, and $\<proc>(\mathrm{act}(u))=\<dest>(\mathrm{act}(v))$.
%%These edges are present in every completion of $t$. By an abuse of notation, for two actions $a$ and $a'$ in $t$, $a\leadsto a'$ (resp., $a\leadsto_1 a'$) denotes the fact that $G'_t$ contains a path (resp., an edge) from the node representing $a$ to that representing $a'$.
%%
%%Let us first assume the case that $G_t'$ is cyclic. By definition, any cycle in $G_t'$ must contain at least one receive action and by Lemma~\ref{lem:acyclic_ag}, at least one of the edges that are not present in $G_t$. Therefore, there exist two actions $\rec{1}{q_1,\_}$ and $\send{2}{p_2,q_1,\_}$, the latter being an unmatched send, such that $\rec{1}{q_1,\_}\leadsto_1 \send{2}{p_2,q_1,\_}$ and $\send{2}{p_2,q_1,\_}\leadsto \rec{1}{q_1,\_}$. 
%%
%%
%%
%%Since we assume well-formed traces, $t$ contains a send action $\send{1}{p_1,q_1,\_}$. Also, since $p_2\neq q_1$, $\send{2}{p_2,q_1,\_}\leadsto \rec{1}{q_1,\_}$ implies that $\send{2}{p_2,q_1,\_}\leadsto \send{1}{p_1,q_1,\_}$.
%%
%%
%%The former implies that there exists a send action $\send{1}{p_1,q_1,\_}$ such that $\send{1}{p_1,q_1,\_}\leadsto \send{2}{p_2,q_1,\_}$ and 
%%
%%TODO THESE PATHS CAN INCLUDE EDGES $\{\send{1}{p_1,q_1,\_},\rec{1}{q_1,\_}\}\rightarrow \{\send{2}{p_3,q_1,\_}\}$ 
%%
%%There are several cases to consider:
%%\begin{itemize}
%Let $\mathrm{act}(v_i)=\{\send{i}{p_i,q_i,\_}, \rec{i}{q_i,\_}\}$, for $i\in [1,2]$. Then, one of the following holds:
%	\begin{itemize}
%		\item $\send{1}{p_1,q_1,\_}\leadsto \send{2}{p_2,q_2,\_}$ and $\rec{2}{q_2,\_}\leadsto \rec{1}{q_1,\_}$, which corresponds to a violation to causal delivery when $q_1=q_2$. Otherwise, when $q_1\neq q_2$, we show that this cycle induces another (smaller cycle) that contains a causal delivery violation. Thus, consider the path between $\rec{2}{q_2,\_}$ and $\rec{1}{q_1,\_}$. By the definition of the conflict relation, the last conflict in this path is necessarily between another action $a$ of $q_1$ (i.e., $\<proc>(a)=q_1$) and $\rec{1}{q_1,\_}$. Otherwise, this path will contain $\send{1}{p_1,q_1,\_}$ and we would have $\send{1}{p_1,q_1,\_}\leadsto  \send{2}{p_2,q_2,\_} \match \rec{2}{q_2,\_}$ and $\rec{2}{q_2,\_}\leadsto \send{1}{p_1,q_1,\_}$ which is impossible by Lemma~\ref{lem:acyclic_ag} (the action graph would be cyclic). Now, the path between $\rec{2}{q_2,\_}$ and $a$ will necessarily contain a conflict due to matching (i.e., a pair of matching send/receive actions); otherwise, $q_1=q_2$. Let $\send{3}{p_3,q_3,\_} \match \rec{3}{q_3,\_}$ be the last such conflict (according to the order in this path). Since it is the last such conflict, we have that $q_3 = q_1$. Therefore, we have $\rec{3}{q_1,\_}\leadsto \rec{1}{q_1,\_}$, and $\send{1}{p_1,q_1,\_}\leadsto \send{2}{p_2,q_2,\_}\match \rec{2}{q_2,\_}\leadsto \send{3}{p_3,q_1,\_}$ which is equivalent to $\send{1}{p_1,q_1,\_}\leadsto \send{3}{p_3,q_1,\_}$. The latter is a violation to causal delivery.
%		
%		\item $\send{1}{p_1,q_1,\_}\leadsto \rec{2}{q_2,\_}$ and $\send{2}{p_2,q_2,\_}\leadsto \rec{1}{q_1,\_}$ which is an exchange pattern.
%		\item $\send{1}{p_1,q_1,\_}\leadsto \rec{2}{q_2,\_}$ and $\rec{2}{q_2,\_}\leadsto \send{1}{p_1,q_1,\_}$ which is impossible by Lemma~\ref{lem:acyclic_ag} because it would imply a cyclic action graph.
%		\item $\rec{1}{q_1,\_}\leadsto \send{2}{p_2,q_2,\_}$ and $\rec{2}{q_2,\_}\leadsto \send{1}{p_1,q_1,\_}$ which is impossible, by Lemma~\ref{lem:acyclic_ag}.
%	\end{itemize}
%%	\item $\mathrm{act}(v_1)=\{\send{1}{p_1,q_1,\_}\}$ and $\mathrm{act}(v_2)=\{\send{2}{p_2,q_2,\_}, \rec{2}{q_2,\_}\}$.
%%\end{itemize}
%%
%%
%%there exist two pairs of matching send/receive actions
%%$\send{1}{p_1,q_1,\_}$, $\rec{1}{q_1,\_}$, and $\send{2}{p_2,q_2,\_}$, $\rec{2}{q_2,\_}$ such that one of the following holds:
%%\begin{itemize}
%%	\item $\send{1}{p_1,q,\_}\leadsto \send{2}{p_2,q,\_}$ and 
%%\end{itemize}
%
%TODO EXTEND TO UNMATCHED TRACES
%\end{proof}

\subsection{Borderline Synchronizability Violations}

For a system $\mathcal{S}$, a violation to $k$-synchronizability $e$ is called \emph{borderline} when every strict prefix of 
$e$ is $k$-synchronizable.

A preliminary result is that every borderline violation $e$ ends with a receive action and this action is included in every cycle of $CG_{tr(e)}$ that is 
bad or exceeds the bound $k$. 

\begin{lemma}
Let $e$ be a borderline violation to $k$-synchronizability of $\mathcal{S}$. Then, $e = e'\cdot r$ for some $e'\in (S_{id}\cup R_{id})^*$ and $r\in R_{id}$.
\end{lemma}
\begin{proof}
Assume by contradiction that $e=e'\cdot s$ for some $e'\in (S_{id}\cup R_{id})^*$ and $s\in S_{id}$. By definition, $CG_{tr(e)}$ contains no outgoing
edge from the node representing $s$, which implies that any cycle of $CG_{tr(e)}$ is already contained in $CG_{tr(e')}$. This is a contradiction to 
the fact that $e$ is borderline.
\end{proof}

Given a cycle $c = v,v_1,\ldots,v_n,v$ of a conflict graph $CG_t$, the node $v$ is called a \emph{critical} node of $c$ when $(v,v_1)$ is an $SX$ edge with $X\in \{S,R\}$ 
and $(v_n,v)$ is an $YR$ edge with $Y\in \{S,R\}$.

\begin{lemma}
Let $e = e'\cdot r$, for some $e'\in (S_{id}\cup R_{id})^*$ and $r\in R_{id}$, be a borderline violation to $k$-synchronizability of $\mathcal{S}$. 
Then, the node $v$ of $CG_{tr(e)}$ representing $r$ (and the corresponding send) is a critical node of every cycle of 
$CG_{tr(e)}$ which is bad or of size bigger than $k$. % is of the form $v, v_1,\ldots, v_n,v$ where $(v,v_1)$ is an $SX$ edge for $X\in \{S,R\}$ and $(v_n,v)$ is an $YR$ edge for $Y\in \{S,R\}$.
\end{lemma}
\begin{proof}
Let $v_0,v_1,\ldots,v_n,v_0$ be a cycle of $CG_{tr(e)}$ which is bad or of size bigger than $k$. We first show that $v=v_i$ for some $0\leq i\leq n$. 
Assume by contradiction that this is not the case. Then, the execution $e'$ is already a violation to $k$-synchronizability which violates the assumption that $e$ is borderline.

For the following, w.l.o.g., we assume that $v=v_0$. Since $r$ is the last action of $e$, the only outgoing edge of $v$ is an edge labeled by $SX$ with $X\in \{S,R\}$. Therefore $(v,v_1)$ is an $SX$ edge. 
Assuming by contradiction that the edge $(v_n,v)$ is labeled by $YS$ for some $Y\in \{S,R\}$ implies that $e'$ is already a $k$-synchronizability violation, which contradicts the hypothesis.
\end{proof}


\subsection{Simulating Borderline Violations on the Synchronous Semantics}\label{ssec:verif1}

 We define a system $\mathcal{S'}$ whose synchronous semantics ``simulates'' a permutation of every borderline violation of 
$\mathcal{S}$, where essentially, the reception of exactly one message is delayed (w.r.t. the synchronous semantics of $\mathcal{S}$).
The reception of a message $m$ sent from $p$ to $q$ is delayed by redirecting it to an additional process $\pi$ which relays it
to $q$ at a later time. 

Formally, given $\mathcal{S}=((\<Lsts>_p,\delta_p,l_p^0)\mid p\in\<Pids>)$, we define $\mathcal{S'}=((\<Lsts>_p,\delta'_p,l_p^0)|p\in\<Pids>\cup\{\pi\})$ where
\begin{itemize}
	\item every send of a process $p$ can be non-deterministically redirected to the process $\pi$ (the message payload contains the destination process), i.e., 
	\begin{align*}
	&\delta'_p(l,\senda{p,\pi,(q,v)}) = \delta'_p(l,\senda{p,q,v})\mbox{, and} \\ 
	&\delta'_p(l,a)=\delta_p(l,a)\mbox{ for all $p\in\<Pids>$, $l\in \<Lsts>_p$, and $a\not\in \{\senda{p,\pi,v}| p\in\<Pids>, v\in\<Vals>\}$}
	\end{align*}
	\item the process $\pi$ stores the received message in its state and relays it later, i.e., $\<Lsts>_\pi=\{l_\pi^0,l_f\}\cup\{(q,v)\mid q\in\<Pids>, v\in\<Vals>\}$, and
	for all $q\in\<Pids>$ and $v\in\<Vals>$, 
	\begin{align*}
	&\delta'_p(l_\pi^0,\reca{\pi,(q,v)}) = \{(q,v)\} \mbox{ and }
	\delta'_p((q,v),\senda{\pi,q,v})=l_f
	\end{align*}	
\end{itemize}

Redirecting messages to $\pi$ may lead to scenarios violating causal delivery (where a message sent later than the redirected one and to the same destination is received
earlier). We show how to exclude such scenarios in Section~\ref{ssec:verif2}.

%We first show that $\synchExec{\mathcal{S'}}{k}$ contains all the borderline violations of $\mathcal{S}$. 




The following result shows that $\synchExec{\mathcal{S'}}{k}$ ``simulates'' all the borderline violations of $\mathcal{S}$, modulo permutations and substitutions of message identifiers. 
%For two executions $e$ and $e'$, $e =_{\mathit{id}} e'$ denotes the fact that $e'$ is obtained from $e$ by renaming message identifiers, i.e., there exists a function $f:\<Mids>->\<Mids>$ such that $e[i]=a_j$, for some $a\in S\cup R$ and $j\in\<Mids>$, iff $e'[i]=a_{f(j)}$ (where $e[i]$ denotes the $i$-th element of the sequence $e$).
%TODO BELONGS TO THIS MODULO CONFLICT-PRESERVING PERMUTATIONS
%For a borderline violation $e=e_1\cdot s\cdot e_2\cdot r$, where $s\match r$, we assume that $e_1\cdot s\cdot e_2$ is $k$-synchronous. By Lemma~\ref{lem:zable_nous}, this is without lost of generality because $e_1\cdot s\cdot e_2$ is $k$-synchronizable.

\begin{theorem}
Let $e=e_1\cdot \send{i}{p,q,v}\cdot e_2\cdot \rec{i}{q,v}$ be a borderline violation to $k$-synchronizability of $\mathcal{S}$. Then, $\synchExec{\mathcal{S'}}{k}$ contains an execution $e'$ of the form (modulo a substitution of message identifiers): 
\begin{align*}
e'=e_1'\cdot \send{i}{p,\pi,(q,v)}\cdot \rec{i}{\pi,(q,v)}\cdot e_2'\cdot \send{j}{\pi,q,v}\cdot \rec{j}{q,v}
\end{align*}
such that $e_1'\cdot \send{i}{p,q,v} \cdot e_2'$ is a permutation of $e_1\cdot \send{i}{p,q,v}\cdot e_2$.
\end{theorem}
\begin{proof}
A direct consequence of the definition of $\mathcal{S'}$ is that $e'\in\asynchExec{\mathcal{S'}}$. We show that the trace of $e'$ is $k$-synchronous. The conflict graph of $tr(e')$ can be obtained from the one of $tr(e)$ as follows:
\begin{itemize}
	\item the node $v$ representing the pair of actions $\{\send{i}{p,q,v},\rec{i}{q,v}\}$ is replaced by two nodes $v'$ and $v''$ representing $\{\send{i}{p,\pi,(q,v)}, \rec{i}{\pi,(q,v)}\}$ and $\{\send{j}{\pi,q,v}, \rec{j}{q,v}\}$, respectively,
	\item for every $SX$ edge from $v$ to a node $w$ in $CG_{tr(e)}$, where $X\in\{S,R\}$, there exists an $SX$ edge from $v'$ to $w$ in $CG_{tr(e')}$,
	\item $v'$ is connected to $v''$ by an $RS$ edge,
	\item there is no outgoing edge from $v''$.
\end{itemize}
Since all the cycles of $CG_{tr(e)}$ that are bad or exceed the size $k$ pass trough $v$, we get that $CG_{tr(e')}$ contains no such cycle.
Therefore, $tr(e')$ is $k$-synchronous. 

This implies that $\synchExec{\mathcal{S'}}{k}$ contains a permutation of $e_1\cdot \send{i}{p,\pi,(q,v)}\cdot \rec{i}{\pi,(q,v)}\cdot e_2\cdot \send{j}{\pi,q,v}\cdot \rec{j}{q,v}$. Since there is no outgoing edge from $v''$, there exists such a permutation that ends in $\send{j}{\pi,q,v}\cdot \rec{j}{q,v}$ which concludes the proof.
\end{proof}

\subsection{Excluding Executions Violating Causal Delivery}\label{ssec:verif2}

\begin{lemma}
Let $e$ be an execution in $\synchExec{\mathcal{S'}}{k}$ of the form 
\begin{align*}
e_1\cdot \send{i}{p,\pi,(q,v)}\cdot \rec{i}{\pi,(q,v)}\cdot e_2\cdot \send{j}{\pi,q,v}\cdot \rec{j}{q,v}.
\end{align*}
If $e$ satisfies causal delivery, then $\asynchExec{\mathcal{S}}$ contains a permutation of $e_1\cdot\send{i}{p,q,v} \cdot e_2\cdot \rec{i}{q,v}$ (where $i$ may be renamed so the execution is well-formed) \end{lemma}

\begin{figure}
\begin{center}
\centering
\begin{lstlisting}
function cone: $2^{\<Pids>}$
function receiver: $\<Pids>\cup \{\bot\}$

// for every $i$, $\<proc>(s_i)\neq \pi$
rule $s_1\cdot\ldots\cdot s_n\cdot r_1\cdot\ldots\cdot r_m$:
  if ( $\exists i,k.\ s_i = \send{k}{p,\pi,(q,v)}$ )
    cone := $\{p\}$
    receiver := $q$
  forall i with $s_i = \send{k}{p,q,v}$
    if ( $p\in \mbox{proc} \land \exists j.\ r_j = \rec{k}{q,v}$ )
      cone := $\mbox{cone} \cup \{q\}$
      assert $q \neq \mbox{receiver}$
\end{lstlisting}
\end{center}
%\begin{lstlisting}
%function firstSend: $\mathbb{B}$
%function proc: $\<Pids>\cup \{\bot\}$
%function receiver: $\<Pids>\cup \{\bot\}$
%function checkPresence: $\<Mids>\cup \{\bot\}$
%
%rule $\send{i}{p,q,v}$:
%  if ( firstSend )
%    assert checkPresence = $\bot$
%    firstSend := false
%  if ( * $\land\ \mbox{proc} = \bot$ )
%    proc := p
%  if ( proc = p )
%    proc := q
%    checkPresence := i
%    
%rule $\rec{i}{q,v}$:
%  choiceEnabled := false
%  if ( checkPresence = i )
%    checkPresence := $\bot$
%  firstSend := true
%\end{lstlisting}
\caption{The monitor $\mathcal{M}_{\mathit{causal}}$. Initially, {\tt proc} and {\tt receiver} are $\bot$.}
\label{fig:mon_causal}
\end{figure}

\begin{lemma}
Let $e$ be an execution in $\synchExec{\mathcal{S'}}{k}$ of the form 
\begin{align*}
e_1\cdot \send{i}{p,\pi,(q,v)}\cdot \rec{i}{\pi,(q,v)}\cdot e_2\cdot \send{j}{\pi,q,v}\cdot \rec{j}{q,v}.
\end{align*}
Then, $e$ satisfies causal delivery iff for every send action $s$
\begin{align*}
(\send{i}{p,\pi,(q,v)}\leadsto_{tr(e)} s\land \<dest>(s)=q)\implies \not\exists r\in e_2.\ s\match r
\end{align*}
\end{lemma}

\begin{lemma}

\end{lemma}

\subsection{Checking for Synchronizability Violation Patterns}

\begin{figure}
\begin{center}
\centering
\begin{lstlisting}
function cycle: $\<Pids>^*$
function receiver: $\<Pids>\cup \{\bot\}$
function lastIsRec: $\mathbb{B}$
function sawRS: $\mathbb{B}$

// for every $i$, $\<proc>(s_i)\neq \pi$
rule $s_1\cdot\ldots\cdot s_n\cdot r_1\cdot\ldots\cdot r_m$:
  if ( $\exists i,k.\ s_i = \send{k}{p,\pi,(q,v)}$ )
    cycle := $p$
    receiver := $q$
  if ( * $\land$ $\exists i,j,k.\ s_i = \send{k}{p,q,v} \land r_j = \rec{k}{q,v}\land \mbox{cycle} \in \<Pids>^*\cdot \{p,q\} $ )
    if ( * )
      cycle := $\mbox{proc}\cdot p$
      if ($\mbox{lastIsRec}\land \mbox{proc} = p$)
        sawRS = true
    else 
      cycle := $\mbox{proc}\cdot q$
      lastIsRec := true

rule $\send{i}{\pi,q,v}\cdot \rec{i}{q,v}$:
  assert $\mbox{cycle} \in \<Pids>^*\cdot\mbox{receiver} \implies (|\mbox{cycle}| \leq k \land \neg \mbox{sawRS})$
\end{lstlisting}
\end{center}
\caption{The monitor $\mathcal{M}_{\mathit{viol}}$. Initially, {\tt proc} and {\tt receiver} are $\bot$.}
\label{fig:mon_viol}
\end{figure}



We define a monitor in the form of a register automaton (a finite state machine equipped with a set of registers)
that 