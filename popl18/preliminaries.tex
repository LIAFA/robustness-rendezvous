%!TEX root = main.tex
\section{Message passing systems}\label{sec:prelims}

%TODO A BUFFER FOR EVERY PAIR OF PROCESSES DOESN'T CHANGE THE RESULTS

We fix arbitrary sets $\<Pids>$ and $\<Val>$ of process ids and message payloads. 
%A \emph{message} is a triple $m = \tup{i,j,p}$ where $i \in \<Ids>$ denotes the source, $j \in \<Ids>$ the destination, and $p \in \<Plds>$ the payload. 
%The set of messages is denoted by $\<Msgs>$. 
%The source, destination, and payload of a message $m$ are denoted by $\<src>(m)$, $\<dst>(m)$, $\<pay>(m)$, respectively.
%We fix an arbitrary set $\<Mids>$ of message identifiers, and the sets
For given sets $\<Ids>$ and $\<Plds>$, we fix the sets 
\begin{align*}
S = \set{\senda{p,q,v}: p,q\in\<Pids>, v\in \<Val>},\mbox{ and }R = \set{\reca{q,v}: q\in\<Pids>, v\in \<Val>}
\end{align*}
of \emph{send actions} and \emph{receive actions}. 
Each send action $\senda{p,q,v}$ combines two process ids $p, q\in \<Ids>$ denoting 
the sender and the receiver of the message, respectively, and a message payload $v\in \<Val>$. Receive actions
specify the process $q\in \<Pids>$ receiving the message, and the message payload $v\in \<Val>$. 
We write $\senda{p,q,\_}$ and $\reca{q,\_}$ when the message payload is unconstrained.
We denote the process executing a send/receive action $a\in S\cup R$ by $\<proc>(a)$, i.e.,
$\<proc>(a)=p$ for every $a=\senda{p,q,v}$, and $\<proc>(a)=q$ for every $a=\reca{q,v}$,
and the destination of a send action $s\in S$ by $\<dest>(s)$, i.e., $\<dest>(\senda{p,q,v})=q$.
%Send and receive actions $s\in S$ and $r\in R$ are \emph{matching}, written $s\match r$, when $\<msg>(s)=\<msg>(r)$.

A message passing system consists of a finite set of processes that communicate via messages. Each process is described as a state
machine that evolves by executing send or receive actions, or local actions (specified as $\epsilon$ transitions)
 and it is equipped with an instance of some fixed collection data type, e.g., FIFO queue, 
that acts as a message buffer (storing incoming messages before being received). 

Formally, a \emph{message passing system} is a tuple $A=(\<Lsts>,\delta,l_0,\<AdtM>)$ where $\<Lsts>$ is a set of local process states,
$\delta\subseteq \<Lsts>\times (S\cup R\cup\{\epsilon\})\times \<Lsts>$ is a transition relation describing the 
evolution of all processes, $l_0$ is the initial state of every process, and $\<AdtM>$ is 
a collection data type  with interface $\mathrm{add}(m)$ for adding an element $m$
to the collection and $\mathrm{rem}()$ that removes and returns an element from the collection (this method returns only when the collection
is non-empty, otherwise it blocks). 
%To simplify the exposition, we assume
%that each receive action is enabled in every local state, i.e., for every $l\in \<Lsts>$ and every $r\in R$, there exists $l'\in \<Lsts>$ such that $(l,r,l')\in \delta$.
%TODO GIVE MORE EXPLANATIONS

A configuration $c=\tup{\vec{l},\vec{b}}$ is a vector of local states $\vec{l}$ together with a vector of message buffers $\vec{b}$ that are
instances of the collection data type $\<AdtM>$. These two vectors are indexed by elements of $\<Pids>$.
For a vector $\vec{x}$, let $\vec{x}_p$ denote the element of $\vec{x}$ of index $p$.
The state of the process $p$ is defined by the local state $\vec{l}_p$ and the message buffer $\vec{b}_p$.
%The set of configurations is denoted by $C$.

\begin{figure} [t]
\footnotesize{
  \centering
  \begin{mathpar}
    \inferrule[send]{
      m= (i,v) \\ 
      i\in \<Mids>\mbox{ fresh} \\
      l\in \delta(\vec{l}_p,\senda{p,q,v}) \neq \emptyset
    }{
      \vec{l},\vec{b}
      \xrightarrow{\send{i}{p,q,v}}
      \vec{l}[\vec{l}_p\gets l],\vec{b}[\vec{b}_q.\ \mathrm{add}(m)]
    }%\hspace{5mm}
    
    \inferrule[receive]{
      (i,v) = \vec{b}_q.\ \mathrm{rem}() \\
      l\in \delta(\vec{l}_q,\reca{q,v}) \neq \emptyset
    }{
      \vec{l},\vec{b}
      \xrightarrow{\rec{i}{q,v}}
      \vec{l}[\vec{l}_q\gets l],\vec{b}[\vec{b}_q.\ \mathrm{rem}()]
    }%\hspace{5mm}
    
    \inferrule[local]{ 
      l\in \delta(\vec{l}_p,\epsilon) \neq \emptyset
    }{
      \vec{l},\vec{b}
      \xrightarrow{\epsilon}
      \vec{l}[\vec{l}_p\gets l],\vec{b}
    }%\hspace{5mm}
  \end{mathpar}
  }
% \vspace{-5mm}
  \caption{The asynchronous semantics of a message passing system $A$. Above, $\vec{l}[\vec{l}_p\gets l]$ denotes an update of $\vec{l}$ where the element of index $p$ is replaced by $l$. Also, $\vec{b}[\vec{b}_q.\ \mathrm{add}(m)]$ denotes the vector of message buffers obtained from $\vec{b}$ by calling the method $\mathrm{add}(m)$ of the element of index $q$.
  }
  \label{fig:asynch-sem}
%\vspace{-6mm}
\end{figure}

We fix an arbitrary set $\<Mids>$ of message identifiers, and the sets 
\begin{align*}
S_{id} = \set{s_i: s\in S, i\in \<Mids>} \mbox{ and } R_{id} = \set{r_i: r\in R, i\in \<Mids>}
\end{align*}
of indexed actions; an indexed action combines a send/receive action with a message identifier $i\in \<Mids>$.
Message identifiers are used to pair send and receive actions.
We denote the message id of an indexed send/receive action $a$ by $\<msg>(a)$.
Indexed send and receive actions $s\in S_{id}$ and $r\in R_{id}$ are \emph{matching}, 
written $s\match r$, when $\<msg>(s)=\<msg>(r)$.
For notational convenience, we often associate $\<Mids>$ with $\<Nats>$, e.g., writing $\send{1}{p,q,v}$ and $\rec{2}{q',v'}$ in place of $\send{i_1}{p,q,v}$ and $\rec{i_2}{q',v'}$.

The transition relation $\rightarrow$ in Figure~\ref{fig:asynch-sem} is determined by a message passing system $A$, and maps
a configuration $c_1$ to another configuration $c_2$ and indexed action $a\in S_{id}\cup R_{id}$ or $\epsilon$.
\textsc{send} transitions ... \textsc{receive} transitions ... \textsc{local} transitions TODO FILL IN

An \emph{execution} of a system $A$ under the asynchronous semantics to configuration ${c}_n$
is a configuration sequence ${c}_0 {c}_1 \ldots {c}_n$ such that
%\vspace{-2mm}
%\begin{align*}
$  {c}_m \xrightarrow{a_{m+1}} {c}_{m+1}$
%\end{align*}
%%\vspace{-1.5mm}
%\noindent
for $0 \le m < n$. We call
the sequence $a_1 \ldots a_n$ the \emph{trace} of ${c}_0 {c}_1 \ldots
{c}_n$ and we say that ${c}_n$ is reachable in $A$ under the asynchronous semantics, with trace $a_1\ldots a_n$. The \emph{reachable local state vectors} of $A$, denoted $\asynch{States}(A)$, is the
set of local state vectors in reachable configurations.
%
The set of traces of $A$ under the asynchronous semantics is denoted by $\asynchTr{A}$.

Given a trace $t$, a send action $s$ in $t$ is called an \emph{unmatched send} when $t$ contains no receive action $r$ with $s\match r$. A trace $t$ is called \emph{matched} when it contains no unmatched send.




