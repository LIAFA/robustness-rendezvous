%!TEX root = main.tex
\section{Message passing systems}\label{sec:prelims}

TODO ASSUME PAYLOADS INCLUDE A MESSAGE IDENTIFIER

We fix arbitrary sets $\<Ids>$ and $\<Plds>$ of process ids and message payloads. 
A \emph{message} is a triple $m = \tup{i,j,p}$ where $i \in \<Ids>$ denotes the source, $j \in \<Ids>$ the destination, and $p \in \<Plds>$ the payload. 
The set of messages is denoted by $\<Msgs>$. 
The source, destination, and payload of a message $m$ are denoted by $\<src>(m)$, $\<dst>(m)$, $\<pay>(m)$, respectively.
For given sets $\<Ids>$ and $\<Plds>$, we fix the sets 
\begin{align*}
S = \set{\send{i}{m}: i\in\<Ids>, m\in \<Msgs>, i=\<src>(m)},\mbox{ and }R = \set{\rec{j}{m}: j\in\<Ids>, m\in \<Msgs>, j=\<dst>(m)}
\end{align*}
of \emph{send actions} and \emph{receive actions}; 
each send action $\send{i}{m}$ combines a process id $i\in \<Ids>$ with a message $m\in \<Msgs>$ whose source is the same process $i$.
We denote the message of a send/receive action $a$ by $\<msg>(a)$, and the process id indexing the action by $\<proc>(a)$.
Send and receive actions $s\in S$ and $r\in R$ are \emph{matching}, written $s\match r$, when $\<msg>(s)=\<msg>(r)$.

A message passing system consists of a finite set of processes that communicate via messages. Each process is described as a state
machine that evolves by executing send or receive actions, or local actions (specified as $\epsilon$ transitions)
 and it is equipped with an instance of some fixed collection data type, e.g., FIFO queue, 
that acts as a message buffer (storing incoming messages before being processed). 

Formally, a \emph{message passing system} is a tuple $A=(\<Lsts>,\delta,l_0,\<AdtM>)$ where $\<Lsts>$ is a set of local process states,
$\delta\subseteq \<Lsts>\times (S\cup R\cup\{\epsilon\})\times \<Lsts>$ is a transition relation describing the 
evolution of all processes, $l_0$ is the initial state of every process, and $\<AdtM>$ is 
a collection data type  with interface $\mathrm{add}(m)$ for adding an element $m$
to the collection and $\mathrm{rem}()$ that removes and returns an element from the collection (this method returns only when the collection
is non-empty). 
To simplify the exposition, we assume
that each receive action is enabled in every local state, i.e., for every $l\in \<Lsts>$ and every $r\in R$, there exists $l'\in \<Lsts>$ such that $(l,r,l')\in \delta$.
TODO GIVE MORE EXPLANATIONS

A configuration $c=\tup{\vec{l},\vec{b}}$ is a vector of local states $\vec{l}$ together with a vector of message buffers $\vec{b}$ that are
instances of the collection data type $\<AdtM>$. For a vector $\vec{x}$, let $\vec{x}_i$ denote the $i$th element of $\vec{x}$.
The state of a process is defined by a local state $\vec{l}_i$ and a message buffer $\vec{b_i}$, for some $i$.
%The set of configurations is denoted by $C$.

\begin{figure} [t]
\footnotesize{
  \centering
  \begin{mathpar}
    \inferrule[send]{
      m= \tup{i,j,p} \\ 
      \delta(\vec{l}_i,\send{i}{m}) \neq \emptyset
    }{
      \vec{l},\vec{b}
      \xrightarrow{\send{i}{m}}
      \vec{l}[\vec{l_i}\gets \delta(\vec{l}_i,\send{i}{m})],\vec{b}[\vec{b}_j.\ \mathrm{add}(m)]
    }%\hspace{5mm}
    
    \inferrule[receive]{
      m = \vec{b}_j.\ \mathrm{rem}() \\
      \delta(\vec{l}_j,\rec{j}{m}) \neq \emptyset
    }{
      \vec{l},\vec{b}
      \xrightarrow{\rec{j}{m}}
      \vec{l}[\vec{l_j}\gets \delta(\vec{l}_j,\rec{j}{m})],\vec{b}[\vec{b}_j.\ \mathrm{rem}()]
    }%\hspace{5mm}
    
    \inferrule[local]{ 
      \delta(\vec{l}_i,\epsilon) \neq \emptyset
    }{
      \vec{l},\vec{b}
      \xrightarrow{\epsilon}
      \vec{l}[\vec{l_i}\gets \delta(\vec{l}_i,\epsilon)],\vec{b}
    }%\hspace{5mm}
  \end{mathpar}
  }
% \vspace{-5mm}
  \caption{The asynchronous semantics of a message passing system $A$. Above, $\vec{l}[\vec{l_i}\gets \delta(\vec{l}_i,\send{i}{m})]$ denotes an update of $\vec{l}$ where the $i$th element is replaced by one of the local states in $\delta(\vec{l}_i,\send{i}{m})$. Also, $\vec{b}[\vec{b}_j.\ \mathrm{add}(m)]$ denotes the vector of message buffers obtained from $\vec{b}$ by calling the method $\mathrm{add}(m)$ of the $j$th element.
  }
  \label{fig:asynch-sem}
%\vspace{-6mm}
\end{figure}


The transition relation $\rightarrow$ in Figure~\ref{fig:asynch-sem} is determined by a message passing system $A$, and maps
a configuration $c_1$ to another configuration $c_2$ and action $a\in S\cup R$.
\textsc{send} transitions ... \textsc{receive} transitions ... \textsc{local} transitions TODO FILL IN

An \emph{execution} of a system $A$ under the asynchronous semantics to configuration ${c}_n$
is a configuration sequence ${c}_0 {c}_1 \ldots {c}_n$ such that
%\vspace{-2mm}
%\begin{align*}
$  {c}_m \xrightarrow{a_{m+1}} {c}_{m+1}$
%\end{align*}
%%\vspace{-1.5mm}
%\noindent
for $0 \le m < n$. We call
the sequence $a_1 \ldots a_n$ the \emph{trace} of ${c}_0 {c}_1 \ldots
{c}_n$ and we say that ${c}_n$ is reachable in $A$ under the asynchronous semantics, with trace $a_1\ldots a_n$. The \emph{reachable local state vectors} of $A$, denoted $\asynch{States}(A)$, is the
set of local state vectors in reachable configurations.
%
The set of traces of $A$ under the asynchronous semantics is denoted by $\asynchTr{A}$.




