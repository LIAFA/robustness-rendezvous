%!TEX root = main.tex
\section{Robustness Against the Synchronous Semantics}

TODO DEFINE CONFLICT-PRESERVING PERMUTATIONS

TODO DEFINE ROBUSTNESS AS SERIALIZABILITY

TODO PROVE THAT IT IMPLIES HAVING THE SAME SET OF LOCAL STATE VECTORS AS IN THE SYNCHRONOUS SEMANTICS

We define robustness against rendez-vous which ensures that even if the sys- tem uses buffers to store messages it provides the illusion that messages are received instantaneously, as in the rendez-vous semantics. This is analogous to the atomicity criterion for concurrent shared-memory systems, where even though transactions can interleave, every execution is “equivalent” to an execution where transactions happen atomically without interference.

\subsection{Defining Robustness}

We define a conflict relation $\prec$ on actions in $S\cup R$ that relates every two actions of the same process and every send with the corresponding receive. Formally,
\begin{align*}
a \prec a'\mbox{ iff } \<proc>(a) = \<proc>(a')\mbox{ or } a\match r
\end{align*}

A permutation $t'$ of a trace $t$ is \emph{conflict-preserving} when every pair $a$ and $a'$ of actions of $t$ appear in the same order in $t'$ whenever $a \prec a'$. Intuitively, a message passing system can not distinguish between two traces, one being the conflict-preserving permutation of the other.

\begin{lemma}
For a given trace $t$, $\asynchTr{A}$ contains every conflict-preserving permutation $t'$ of $t$ when $t\in \asynchTr{A}$.
\end{lemma}

A trace $t$ is called \emph{synchronous} when every receive $r\in R$ is immediately preceded by the matching send. Formally, for every $1\leq i\leq |t|$, if $t_i\in R$ then $t_{i-1}\in S$ and $t_{i-1}\match t_i$.

\begin{definition}
A trace $t$ is called \emph{synchronizable} when there exists a synchronous conflict-preserving permutation of $t$. 
\end{definition}

\begin{figure} [t]
\footnotesize{
  \centering
  \begin{mathpar}
    \inferrule[send]{
      m= \tup{i,j,p}
    }{
      \vec{l},\vec{b}
      \xRightarrow{\send{i}{m}}
      \vec{l}[\vec{l_i}\gets T(\vec{l}_i,\send{i}{m})],\vec{b}[\vec{b}_j.\ \mathrm{add}(m)]
    }%\hspace{5mm}
    
    \inferrule[receive]{
      m = \vec{b}_j.\ \mathrm{rem}() \\
      \vec{b_i}. \mathrm{rem}() \mbox{ is not enabled, for all $i\neq j$}
    }{
      \vec{l},\vec{b}
      \xRightarrow{\rec{j}{m}}
      \vec{l}[\vec{l_j}\gets T(\vec{l}_j,\rec{j}{m})],\vec{b}[\vec{b}_j.\ \mathrm{rem}()]
    }%\hspace{5mm}
  \end{mathpar}
  }
% \vspace{-5mm}
  \caption{The synchronous semantics of a message passing system $A$. 
  }
  \label{fig:synch-sem}
%\vspace{-6mm}
\end{figure}

