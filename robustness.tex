%!TEX root = main.tex
\section{Synchronizability}

We define a notion of robustness, called \emph{synchronizability}, which ensures that even if the system uses buffers to store messages it provides the illusion that messages are received instantaneously, as in a synchronous (rendez-vous) semantics. This is analogous to the atomicity criterion for concurrent shared-memory systems, where even though transactions can interleave, every execution is “equivalent” to an execution where transactions happen atomically without interference.

%\subsection{Defining Robustness}

We define a conflict relation $\prec$ on actions in $S_{id}\cup R_{id}$ that relates every two actions of the same process and every send with the corresponding receive. Formally,
\begin{align*}
a \prec a'\mbox{ iff } \<proc>(a) = \<proc>(a')\mbox{ or } a\match a'
\end{align*}

A permutation $t'$ of a trace $t$ is \emph{conflict-preserving} when every pair $a$ and $a'$ of actions of $t$ appear in the same order in $t'$ whenever $a \prec a'$. 
From now on, whenever we use the term permutation we mean conflict-preserving permutation.
Intuitively, a message passing system can not distinguish between two traces, one being a conflict-preserving permutation of the other.

\begin{example}\label{ex:perm}
The following traces 
\begin{align*}
\send{1}{p_1,q,\_}\ 
\send{2}{p_2,q,\_}\ 
\rec{1}{q,\_}\ 
\rec{2}{q,\_} \\
\send{1}{p_1,q_1,\_}\ 
\send{2}{p_2,q_2,\_}\ 
\rec{2}{q_2,\_}\ 
\rec{1}{q_1,\_} 
\end{align*}
have the following conflict-preserving permutations, respectively:
\begin{align*}
\send{1}{p_1,q_1,\_}\ 
\rec{1}{q,\_}\ 
\send{2}{p_2,q,\_}\ 
\rec{2}{q,\_} \\
\send{1}{p_1,q_1,\_}\ 
\rec{1}{q_1,\_}\ 
\send{2}{p_2,q_2,\_}\ 
\rec{2}{q_2,\_}\ 
\end{align*}
\end{example}

\begin{lemma}
For a given trace $t$, $\asynchTr{A}$ contains every conflict-preserving permutation $t'$ of $t$ when $t\in \asynchTr{A}$.
\end{lemma}

A trace $t$ is called \emph{synchronous} when every send $s\in S_{id}$ is immediately followed by the matching receive. Formally, for every $0\leq k< |t| -1$, if $t_k\in S_{id}$ then $t_{k+1}\in R_{id}$ and $t_{k}\match t_{k+1}$.

Informally, a trace  $t$ is called \emph{synchronizable} when there exists a synchronous conflict-preserving permutation of $t$. 
However, the presence of send actions without a matching receive introduces a subtlety: a trace is considered synchronizable by supposing that every message is eventually received.
To account for this, we define a \emph{completion} of a trace $t$ to be any sequence $t\cdot t'$ where $t'$ contains only receive actions that match one of the unmatched send actions in $t$.

\begin{definition}
A trace $t$ is called \emph{synchronizable} when there exists a synchronous conflict-preserving permutation of a completion of $t$. 
\end{definition}

\begin{example}
The traces in Example~\ref{ex:perm} are synchronizable while the following are not:
\begin{align*}
\send{1}{p,q,\_}\ 
\send{2}{p,q,\_}\ 
\rec{2}{q,\_}\ 
\rec{1}{q,\_} \\
\send{1}{p,q,\_}\ 
\send{2}{q,p,\_}\ 
\rec{2}{p,\_}\ 
\rec{1}{q,\_} 
\end{align*}
\end{example}

A message passing system $A$ is synchronizable when every trace $t\in \asynchTr{A}$ is  synchronizable.

\begin{figure} [t]
\footnotesize{
  \centering
  \begin{mathpar}
    \inferrule[send]{
      l\in \delta(\vec{l}_p,\senda{p,q,v}) \neq \emptyset \\
      \mathrm{empty}(\vec{b}_j) \mbox{ for all $j$} 
    }{
      \vec{l},\vec{b}
      \xRightarrow{\senda{p,q,v}}
      \vec{l}[\vec{l}_p\gets l],\vec{b}[\vec{b}_q.\ \mathrm{add}(v)]
    }%\hspace{5mm}

%    \inferrule[orphan send]{
%      m= \tup{i,j,p} \\
%      \delta(\vec{l}_i,\send{i}{m})\neq \emptyset
%    }{
%      \vec{l},\vec{b}
%      \xRightarrow{\send{i}{m}}
%      \vec{l}[\vec{l_i}\gets \delta(\vec{l}_i,\send{i}{m})],\vec{b}
%    }%\hspace{5mm}
    
        \inferrule[local]{ 
      l \in \delta(\vec{l}_p,\epsilon) \neq \emptyset
    }{
      \vec{l},\vec{b}
      \xrightarrow{\epsilon}
      \vec{l}[\vec{l}_p\gets l],\vec{b}
    }%\hspace{5mm}
    
    \inferrule[receive]{
      v = \vec{b}_q.\ \mathrm{rem}() \\
      l\in \delta(\vec{l}_q,\reca{q,v}) \neq \emptyset \\
      \mathrm{empty}(\vec{b}_j) \mbox{ for all $j\neq q$} 
    }{
      \vec{l},\vec{b}
      \xRightarrow{\reca{q,v}}
      \vec{l}[\vec{l}_q\gets l],\vec{b}[\vec{b}_q.\ \mathrm{rem}()]
    }%\hspace{5mm}
  \end{mathpar}
  }
% \vspace{-5mm}
  \caption{The synchronous semantics of a message passing system $A$. The predicate $\mathrm{empty}(\vec{b}_j)$ denotes the fact that the buffer $\vec{b}_j$ is empty, i.e, $\vec{b}_j.\ \mathrm{rem}()$ is not enabled.}
  \label{fig:synch-sem}
%\vspace{-6mm}
\end{figure}

The transition relation $\Rightarrow$ in Figure~\ref{fig:asynch-sem} defines a new semantics for  a message passing system $A$
where a receive for process $q$ is enabled only if the buffers of all the other processes are empty. Executions and reachable local
state vectors under the new semantics, denoted by $\synch{States}(A)$, are defined as in Section~\ref{sec:prelims}.
The set of traces of $A$ under the new semantics is denoted by $\synch{Tr}(A)$.

TODO DESCRIBE TRANSITIONS 

\begin{lemma}\label{lem:sync_traces}
The synchronous semantics of a message passing system admits every synchronous trace.
\end{lemma}

TODO DEFINE THE CONDITION XXX THAT EVERY MESSAGE IS RECEIVED

\begin{lemma}
For a given synchronizable XXX message passing system $A$, $\asynch{States}(A)=\synch{States}(A)$.
\end{lemma}
\begin{proof}
TODO 
\end{proof}

\section{Monitoring Synchronizability}

Checking that a given trace is synchronizable can be done by tracking a ``conflict-graph'' like in conflict serializability~\cite{}
to show that every pair of matching send and receive actions can be executed atomically. Roughly, the nodes of this graph
correspond to such pairs of matching actions, and the edges to conflicts between actions from one pair and another. 
Then, a trace is synchronizable if{f} the conflict-graph is acyclic. Dealing with unmatched sends and completions 
poses a technical difficulty since the conflict-graph should contain enough additional edges
to anticipate for the possible completions. 

Before defining conflict-graphs, we define a notion of action-graph that represents the order between all conflicting actions in some given trace.

\begin{definition}[Action-Graph]\label{def:pr_graphs}
    The \emph{action-graph} of a trace $t$ is the directed graph 
    $G_t = \tup{V,E}$ where there is a node in $V$ for each action in $t$, and $E$ 
    contains an edge from $u$ to $v$ iff $\mathrm{act}(u) \prec \mathrm{act}(v)$ and $\mathrm{act}(u)$ occurs before $\mathrm{act}(v)$ in $t$ (where $\mathrm{act}(v)$ is the action of trace $t$ corresponding to the graph node $v$).
\end{definition}
%Intuitively, one can think of the action-graph of a trace $t$ as a structure that represents the order between all conflicting actions in $t$.
%AND THE WEAKEST CONSTRAINTS THAT CAN BE ADDED BY A COMPLETION.
For two actions $a$ and $a'$ in a trace $t$, $a\leadsto a'$ denotes the fact that $G_t$ contains a path from the node representing $a$ to that representing $a'$.

\begin{lemma}\label{lem:acyclic_ag}
	The \emph{action-graph} of a trace is acyclic. Moreover, $\rec{1}{q_1,\_}\leadsto \send{2}{p_2,q_2,\_}$ and $\rec{2}{q_2,\_}\leadsto \send{1}{p_1,q_1,\_}$ is impossible for any possible instantiation of $p_1$, $q_1$, $p_2$, and $q_2$.
\end{lemma}

%TODO BELOW WE ADD EDGES FROM A SEND/RECEIVE PAIR TO EVERY UNMATCHED SEND THAT HAS THE SAME DESTINATION

\begin{definition}[Conflict-Graph]\label{def:conf_graph}
    The \emph{conflict-graph} of a trace $t$ is the directed graph $CG_t=\tup{V',E'}$ where $V'$ includes one node for each pair of matching send and receive actions, and each unmatched send action, in $t$, and we have $(v,v') \in E'$ iff one of the following holds
\begin{enumerate}
    	\item there exist actions $a \in \mathrm{act}(v)$ and $a' \in \mathrm{act}(v')$ such that $(a,a') \in E$ where $G_t = (V,E)$ is the action-graph of $t$ (and $\mathrm{act}(v)$ is the set of actions of trace $t$ corresponding to the graph node $v$), or
	\item\label{def:conf_graph:item2} $\mathrm{act}(v)$ consists of a pair of matching send action $s_1$ and receive action $r_1$, and $\mathrm{act}(v')$ consists of only a send action $s_2$ with $\<dest>(s_1)=\<dest>(s_2)$.
\end{enumerate}
\end{definition}

The fact that a matched trace is synchronizable whenever its conflict-graph is acyclic is a direct consequence of previous results on conflict serializability~\cite{} (in this case, item~\ref{def:conf_graph:item2} of Definition~\ref{def:conf_graph} doesn't apply).

\begin{lemma}\label{lem:cg_matched}
A matched trace is synchronizable if{f} its conflict-graph is acyclic.
\end{lemma}

To extend this result to traces with unmatched sends, we first consider a notion of canonical completion that is enough to recognize all synchronizable traces.
Thus, a \emph{completion} $t\cdot t'$ of a trace $t$ is called \emph{canonical} when for every two unmatched send actions $\send{1}{p_1,q_1,\_}$ and $\send{2}{p_2,q_2,\_}$ occurring in this order in $t$, the suffix $t'$ contains $\rec{1}{q_1,\_}$ before $\rec{2}{q_2,\_}$.

\begin{lemma}\label{lem:synch_canonical}
A trace $t$ is synchronizable if{f} the canonical completion of $t$ is synchronizable.
\end{lemma}
\begin{proof}
The ``if'' direction is trivial. For the reverse, let $t$ be a synchronizable trace. Therefore, there exists a completion $t\cdot t'$ which is synchronizable. If $t\cdot t'$ is canonical, then the proof is finished. Otherwise, let $\send{1}{p_1,q_1,\_}$ and $\send{2}{p_2,q_2,\_}$ be two unmatched send actions occurring in this order in $t$ such that $\rec{2}{q_2,\_}$ occurs before $\rec{1}{q_1,\_}$ in $t'$. Note that, since $\send{1}{p_1,q_1,\_}$ occurs before $\send{2}{p_2,q_2,\_}$, it is impossible that $\send{2}{p_2,q_2,\_} \leadsto \send{1}{p_1,q_1,\_}$.
By Lemma~\ref{lem:cg_matched}, the conflict graph of $t\cdot t'$ is acyclic. 
Let $t\cdot t''$ be a permutation of $t\cdot t'$ where $\rec{2}{q_2,\_}$ and $\rec{1}{q_1,\_}$ are swapped (i.e., $\rec{1}{q_1,\_}$ occurs before $\rec{2}{q_2,\_}$). If $q_1\neq q_2$, then this permutation is conflict-preserving and the conflict graph of $t\cdot t''$ is the same as the one of $t\cdot t'$. 
Now, assume that $q_1=q_2$. Then, the conflict-graph of $t\cdot t''$ is the conflict-graph of $t\cdot t'$ where the edge from the node $u$ representing $\{\send{2}{p_2,q_2,\_},\rec{2}{q_2,\_}\}$ to the node $v$ representing $\{\send{1}{p_1,q_1,\_}, \rec{1}{q_1,\_}\}$ is replaced with an edge from $v$ to $u$.
Since the conflict-graph of $t\cdot t'$ is acyclic, the action graph of $t\cdot t'$ contains no path from $\send{1}{p_1,q_1,\_}$ to $\send{2}{p_2,q_2,\_}$ (otherwise, this path together with the edge from $\rec{2}{q_2,\_}$ to $\rec{1}{q_1,\_}$ will induce a conflict-graph cycle). Therefore, the only path in $CG_{t\cdot t'}$ between $u$ and $v$ is the single-edge path from $u$ to $v$, and changing the direction of this edge will result in a graph that remains acyclic. Consequently, the conflict-graph of $t\cdot t''$ is acyclic. Applying this reasoning further for every pair of unmatched send operations in $t$ shows that the canonical completion of $t$ is synchronizable.
\end{proof}


\begin{theorem}\label{thm:cg} 
A trace $t$ is synchronizable if{f} $CG_t$ is acyclic.
\end{theorem}
\begin{proof}
Let $t$ be a synchronizable trace. Then, there exists a completion $t\cdot t'$ of $t$ such that $CG_{t\cdot t'}$ is acyclic. Note that $CG_{t}$ is a subgraph of $CG_{t\cdot t'}$, which implies that $CG_{t}$ is acyclic as well.

For the reverse, let $t$ be a trace with an acyclic conflict graph. Let $t\cdot t'$ be the canonical completion of $t$. We show that $CG_{t\cdot t'}$ is acyclic which implies by Lemma~\ref{lem:synch_canonical}, that $t$ is synchronizable. Note that $CG_{t}$ is a subgraph of $CG_{t\cdot t'}$. Also, the edges that are in $CG_{t\cdot t'}$ and not in $CG_{t}$ are of the form $(v,v')$ where 
\begin{itemize}
	\item $\mathrm{act}(v)=\{\send{1}{p_1,q,\_}\}$, $\send{2}{p_2,q,\_}\in \mathrm{act}(v')$, and $\send{1}{p_1,q,\_}$ occurs before $\send{2}{p_2,q,\_}$ in $t$, for any $p_1\neq p_2$ (the set $\mathrm{act}(v')$ may or not contain a matching receive).
\end{itemize}
These edges come from the conflicting actions $\rec{1}{q,\_}$ and $\rec{2}{q,\_}$ occurring in this order in $t'$. Note that $(v,v')$ would be already an edge of $CG_{t}$ if $p_1=p_2$. 

Since $\send{1}{p_1,q,\_}$ occurs before $\send{2}{p_2,q,\_}$ in $t$, $CG_{t}$ contains no path from $v'$ to $v$. Therefore, these edges of $CG_{t\cdot t'}$ do not create any cycles (together with the paths in  $CG_{t}$).
\end{proof}



