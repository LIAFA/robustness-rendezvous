%!TEX root = main.tex
\section{Checking Synchronizability}


Verifying a (bounded state) message passing system using a monitor that maintains the conflict-graph of a trace 
is undecidable since it requires dealing with known undecidable features, e.g., FIFO queues.
We show in this section that checking synchronizability can be reduced 
to a reachability problem in a system that executes under the synchronous semantics, and 
thus doesn't use unbounded message buffers. This result is based on the following ideas
\begin{itemize}
	\item consider a class of boarderline violations to synchronizability, i.e., traces which
are not synchronizable and every strict prefix is synchronizable.
	\item starting from the original system $A$, define a new system $A_0$ executing under
the synchronous semantics, which simulates minimal synchronizability violations of
$A$, if any, and $A_0$ goes to an error state whenever such a violation exists.
\end{itemize}

\subsection{Synchronizability Violation Patterns}


\begin{figure}[t]
Causal delivery violation:
\begin{align*}
\send{1}{p_1,q,\_}\leadsto \send{2}{p_2,q,\_}\mbox{ and }
  \rec{1}{q,\_} \not \leadsto \rec{2}{q,\_}  
\end{align*}
Exchange pattern:
\begin{align*}
\send{1}{p_1,q_1,\_}\leadsto \rec{2}{q_2,\_}\mbox{ and }
\send{2}{p_2,q_2,\_}\leadsto \rec{1}{q_1,\_}
\end{align*}
\caption{Synchronizability violation patterns. The predicate $\rec{1}{q,\_} \not \leadsto \rec{2}{q,\_} $ holds either when the trace contains no action $\rec{1}{q,\_}$, or when this action occurs after $\rec{2}{q,\_} $.}
\label{fig:patterns}
\end{figure}

We show that any trace violating synchronizability contains one of the two violation patterns in Figure~\ref{fig:patterns}.
Intuitively, the first pattern describes a violation to \emph{causal delivery} where two causally related messages sent to the same process
are received in an order different from that in which they were sent. Here, the causal order between messages is defined
by the paths in the action graph. The second violation pattern describes a situation in which two messages are sent concurrently 
and each message is received after the other one is sent. This situation arises for instance in protocols solving consensus which alternate
between phases in which all processes send messages to their peers and phases in which they receive messages from their peers. 
In these cases, the sends and receives in the exchange pattern are causally related since they are executed by the same process.

We say that a trace contains a causal delivery violation or an exchange pattern if it contains the actions and the action-graph paths
specified in Figure~\ref{fig:patterns}.

\begin{theorem}\label{th:patt}
A trace is not synchronizable if{f} it contains a causal delivery violation or an exchange pattern.
\end{theorem}
\begin{proof}
The ``only if'' direction is trivial. For both violation patterns, the two pairs of matching send/receive actions (or one pair and an unmatched send, in the case of the causal delivery violations) will define a conflict graph cycle.

For the ``if'' direction, let $t$ be a non-synchronizable trace. Then, the conflict graph of $t$ contains two nodes $v_1$ and $v_2$ such that there is a path from $v_1$ to $v_2$ and vice-versa. 
%By an abuse of notation, assume that the action graph contains also edges 
%$v_1\leadsto v_2$ and $v_2\leadsto v_1$.
Let $G_t'$ be the action graph of $t$ extended with edges between every two nodes $u$ and $v$ such that 
$\mathrm{act}(u)\in R_{id}$, $\mathrm{act}(v)\in S_{id}$ is an unmatched send, and $\<proc>(\mathrm{act}(u))=\<dest>(\mathrm{act}(v))$.
These edges are present in every completion of $t$. By an abuse of notation, for two actions $a$ and $a'$ in $t$, $a\leadsto a'$ (resp., $a\leadsto_1 a'$) denotes the fact that $G'_t$ contains a path (resp., an edge) from the node representing $a$ to that representing $a'$.

Let us first assume the case that $G_t'$ is cyclic. By definition, any cycle in $G_t'$ must contain at least one receive action and by Lemma~\ref{lem:acyclic_ag}, at least one of the edges that are not present in $G_t$. Therefore, there exist two actions $\rec{1}{q_1,\_}$ and $\send{2}{p_2,q_1,\_}$, the latter being an unmatched send, such that $\rec{1}{q_1,\_}\leadsto_1 \send{2}{p_2,q_1,\_}$ and $\send{2}{p_2,q_1,\_}\leadsto \rec{1}{q_1,\_}$. 



Since we assume well-formed traces, $t$ contains a send action $\send{1}{p_1,q_1,\_}$. Also, since $p_2\neq q_1$, $\send{2}{p_2,q_1,\_}\leadsto \rec{1}{q_1,\_}$ implies that $\send{2}{p_2,q_1,\_}\leadsto \send{1}{p_1,q_1,\_}$.


The former implies that there exists a send action $\send{1}{p_1,q_1,\_}$ such that $\send{1}{p_1,q_1,\_}\leadsto \send{2}{p_2,q_1,\_}$ and 

TODO THESE PATHS CAN INCLUDE EDGES $\{\send{1}{p_1,q_1,\_},\rec{1}{q_1,\_}\}\rightarrow \{\send{2}{p_3,q_1,\_}\}$ 

There are several cases to consider:
\begin{itemize}
	\item $\mathrm{act}(v_i)=\{\send{i}{p_i,q_i,\_}, \rec{i}{q_i,\_}\}$, for $i\in [1,2]$. Then, one of the following holds:
	\begin{itemize}
		\item $\send{1}{p_1,q_1,\_}\leadsto \send{2}{p_2,q_2,\_}$ and $\rec{2}{q_2,\_}\leadsto \rec{1}{q_1,\_}$, which corresponds to a violation to causal delivery when $q_1=q_2$. Otherwise, when $q_1\neq q_2$, we show that this cycle induces another (smaller cycle) that contains a causal delivery violation. Thus, consider the path between $\rec{2}{q_2,\_}$ and $\rec{1}{q_1,\_}$. By the definition of the conflict relation, the last conflict in this path is necessarily between another action $a$ of $q_1$ (i.e., $\<proc>(a)=q_1$) and $\rec{1}{q_1,\_}$. Otherwise, this path will contain $\send{1}{p_1,q_1,\_}$ and we would have $\send{1}{p_1,q_1,\_}\leadsto  \send{2}{p_2,q_2,\_} \match \rec{2}{q_2,\_}$ and $\rec{2}{q_2,\_}\leadsto \send{1}{p_1,q_1,\_}$ which is impossible by Lemma~\ref{lem:acyclic_ag} (the action graph would be cyclic). Now, the path between $\rec{2}{q_2,\_}$ and $a$ will necessarily contain a conflict due to matching (i.e., a pair of matching send/receive actions); otherwise, $q_1=q_2$. Let $\send{3}{p_3,q_3,\_} \match \rec{3}{q_3,\_}$ be the last such conflict (according to the order in this path). Since it is the last such conflict, we have that $q_3 = q_1$. Therefore, we have $\rec{3}{q_1,\_}\leadsto \rec{1}{q_1,\_}$, and $\send{1}{p_1,q_1,\_}\leadsto \send{2}{p_2,q_2,\_}\match \rec{2}{q_2,\_}\leadsto \send{3}{p_3,q_1,\_}$ which is equivalent to $\send{1}{p_1,q_1,\_}\leadsto \send{3}{p_3,q_1,\_}$. The latter is a violation to causal delivery.
		
		\item $\send{1}{p_1,q_1,\_}\leadsto \rec{2}{q_2,\_}$ and $\send{2}{p_2,q_2,\_}\leadsto \rec{1}{q_1,\_}$ which is an exchange pattern.
		\item $\send{1}{p_1,q_1,\_}\leadsto \rec{2}{q_2,\_}$ and $\rec{2}{q_2,\_}\leadsto \send{1}{p_1,q_1,\_}$ which is impossible by Lemma~\ref{lem:acyclic_ag} because it would imply a cyclic action graph.
		\item $\rec{1}{q_1,\_}\leadsto \send{2}{p_2,q_2,\_}$ and $\rec{2}{q_2,\_}\leadsto \send{1}{p_1,q_1,\_}$ which is impossible, by Lemma~\ref{lem:acyclic_ag}.
	\end{itemize}
	\item $\mathrm{act}(v_1)=\{\send{1}{p_1,q_1,\_}\}$ and $\mathrm{act}(v_2)=\{\send{2}{p_2,q_2,\_}, \rec{2}{q_2,\_}\}$.
\end{itemize}


there exist two pairs of matching send/receive actions
$\send{1}{p_1,q_1,\_}$, $\rec{1}{q_1,\_}$, and $\send{2}{p_2,q_2,\_}$, $\rec{2}{q_2,\_}$ such that one of the following holds:
\begin{itemize}
	\item $\send{1}{p_1,q,\_}\leadsto \send{2}{p_2,q,\_}$ and 
\end{itemize}
\end{proof}

\subsection{Simulating Borderline Violations}

A violation to synchronizability $t$ is called \emph{borderline} when every strict prefix of $t$ is synchronizable.
We first show that any borderline synchronizability violation of a message passing system $A$ can be simulated by 
the synchronous semantics of another message passing system $B$ computable in polynomial time.

Let $t$ be a borderline synchronizability violation.
By Theorem~\ref{th:patt}, $t$ contains either a causal delivery violation or an exchange pattern.
Therefore, we have 
\begin{align*}
t=t_1\cdot \send{1}{p_1,q_1,\_} \cdot t_2\cdot \rec{1}{q_1,\_}
\end{align*} 
where the prefix $t' = t_1\cdot \send{1}{p_1,q_1,\_} \cdot t_2$ is a synchronizable trace.
We first define a system $\mathrm{lossy}(A)$ whose synchronous semantics simulates every synchronizable 
trace of $A$, and then the system $B$ which simulates the complete trace $t$. The meaning of 
the term ``simulation'' is defined later on.

A trace $t$ is called \emph{synchronous but lossy} when every receive $r\in R_{id}$ is immediately 
preceded by the matching send. Formally, for every $0 < k\leq |t| -1$, if $t_k\in R_{id}$ then 
$t_{k-1}\in S_{id}$ and $t_{k-1}\match t_{k}$.

\begin{lemma}
A trace $t$ is synchronizable whenever there exists a synchronous but lossy permutation of $t$.
\end{lemma}

A synchronous but lossy trace $t$ can be rewritten to a synchronous trace by redirecting all unreceived messages 
to a fixed additional process $\pi_0$. Let $\mathrm{remove\text{-}lossiness}(t)$ be the trace obtained from $t$
by replacing every unmatched $\send{i}{p,q,v}$ with $\send{i}{p,\pi_0,v}\cdot \rec{i}{\pi_0,v}$.

We define the system $\mathrm{lossy}(A)$ whose 

 admits $\mathrm{remove\text{-}lossiness}(t')$ 


By Lemma~\ref{lem:sync_traces}, the synchronous semantics of $A$ admits all synchronous traces. 
However, it may not admit any conflict-preserving permutation 


synchronizable traces are not admitted by the synchronous semantics because some of the sent messages
are never received, i.e., such traces contain send actions without matching receives. 
However, they are admitted by the synchronous semantics of another system $\mathrm{lossy}(A)$ where
messages can be non-deterministically redirected to a fixed additional process $\pi_0$.
More precisely, given a synchronizable trace

$\synch{Tr}(\mathrm{lossy}(A))$ contains a 


Thus, if $A=(\<Lsts>,\delta,l_0,\<AdtM>)$
then $\mathrm{lossy}(A)=(\<Lsts>,\delta',l_0,\<AdtM>)$ where
\begin{align*}
\delta'=\delta\cup\set{(l,\senda{p,\pi_0,v},l'),(l_0,\reca{\pi_0,v},l_0) : \exists\, \senda{p,q,v}\in S.\ (l,\senda{p,q,v},l')\in \delta}.
\end{align*}
In words, every send transition to some process $q$ is cloned to a send transition to process $\pi_0$ 
together with a corresponding receive transition of process $\pi_0$.

\begin{lemma}
fsdfsf
\end{lemma}

To deal with the last receive action, the trace $t$ can be transformed into a synchronous trace $t'$ by redirecting the message $\tup{i_1,j_1,v_1}$ to
a distinguished process $p$ that relays it to $j_1$ later, i.e.,
\begin{align*}
t'=t_1\cdot \send{i_1}{i_1,p,v_1}\cdot \rec{p}{i_1,p,v_1} \cdot t_2\cdot \send{p}{p,j_1,v_1}\cdot \rec{j_1}{p,j_1,v_1}
\end{align*} 

\begin{lemma}
The trace $t'$ is a synchronous trace when $t$ is a borderline violation.
\end{lemma}

In the following, we define a message passing system $B$ whose synchronous semantics admits $t'$ whenever 
$A$ admits $t$ (under the asynchronous semantics).

By Lemma~\ref{lem:sync_traces}, the prefix $t'$ is admitted by the 
synchronous semantics of $A$, i.e., $t'\in \synch{Tr}(A)$.

%may contain send actions without matching receives and it is not admitted as it is by the 
%synchronous semantics of $A$, i.e., $t'\not\in \synch{Tr}(A)$.
%We show however that the configurations reached in $A$ by an execution with trace $t'$ can be reached
%in another message passing system $A'$ even under the synchronous semantics. 
%
%Thus, $A'$ is obtained from $A$ by non-deterministically redirecting messages to some distinguished process $p$
%that accepts any message.
%Formally, if $A=(\<Lsts>,\delta,l_0,\<AdtM>)$
%then $A'=(\<Lsts>\uplus\{l_p\},\delta_2,l_0,\<AdtM>)$ where
%\begin{itemize}
%	\item every send transition to some process $j$ is cloned to a send transition to process $p$, i.e.,
%\begin{align*}
%\delta_1=\delta\cup\set{(l,\send{i}{i,p,v},l'): \exists\, \send{i}{i,j,v}\in S.\ (l,\send{i}{i,j,v},l')\in \delta}.
%\end{align*}
%	\item the process $p$ can receive any message
%\begin{align*}
%\delta_2=\delta_1\cup\set{(l_p,\rec{p}{i,p,v},l_p): \exists\, \send{i}{i,p,v}\in S.\ (l,\send{i}{i,p,v},l')\in \delta_1}.
%\end{align*}
%	
%\end{itemize}
%For a given synchronous trace $t$, let $\mathrm{redirect}\text{-}\mathrm{orphans}(t)$ be the trace obtained from $t$ by 
%replacing every unmatched send action $\send{i}{i,j,v}$ with $\send{i}{i,p,v}\rec{p}{i,p,v}$.
%
%TODO REDIRECT ORPHANS PRESERVES ACYCLICITY PRECISELY


%\begin{lemma}
%Let $(\vec{l},\vec{b})$ be a configuration reachable in $A$ with a synchronous trace $t'$ (under the asynchronous semantics). 
%Then, $(\vec{l},\vec{\emptyset})$ is reachable in $A'$ with trace $\mathrm{redirect}\text{-}\mathrm{orphans}(t')$. 
%\end{lemma}

%To simulate the last receive in $t$ we transform $A$ such that any message sent to some process $j_1$ can be redirected to some 
%distinguished process $p$ who relays it to $j_1$ non-deterministically at a later time. We constrain the system such that it can
%redirect only one message. Formally, 
Thus, if $A=(\<Lsts>,\delta,l_0,\<AdtM>)$
then $B=(\<Lsts>\uplus\set{l_j: j\in\<Plds>},\delta_2,l_0,\<AdtM>)$ where
\begin{itemize}
	\item every send transition to some process $j$ is cloned to a send transition to process $p$ 
and a corresponding receive transition of process $p$ (the local state of $p$ stores the identifier of process $j$), i.e.,
\begin{align*}
\delta_1=\delta\cup\set{(l,\send{i}{i,p,v},l),(l_0,\rec{p}{i,p,v},l_{j}) : \exists\, \send{i}{i,j,v}\in S.\ (l,\send{i}{i,j,v},l')\in \delta}.
\end{align*}
%
%	
%\begin{align*}
%\delta_1=\delta\cup\set{(l,\send{i}{i,p,(j,v)},l'): \exists\, \send{i}{i,j,v}\in S.\ (l,\send{i}{i,j,v},l')\in \delta}.
%\end{align*}
%	\item the process $p$ can receive any message, i.e.,
%\begin{align*}
%\delta_2=\delta_1\cup\{(l_0,\epsilon,l_p)\}\cup \set{(l_p,\rec{p}{i,p,v},l_p): \exists\, \send{i}{i,p,v}\in S.\ (l,\send{i}{i,p,v},l')\in \delta_1}.
%\end{align*}
	\item every receive transition of some process $j$ is cloned to a send transition of process $p$ 
and a corresponding receive transition from process $p$, i.e.,
\begin{align*}
\delta_2=\delta_1\cup\set{(l_j,\send{p}{p,j,v},l_0),(l,\rec{j}{p,j,v},l'): \exists\, \rec{j}{i,j,v}\in R.\ (l,\rec{j}{i,j,v},l')\in \delta}.
\end{align*}
\end{itemize}

TODO CAN WE HAVE PROBLEMS IF THE $i$s IN THE SEND AND RECEIVE TRANSITIONS ARE NOT THE SAME. MAYBE WE SHOULD REMOVE $i$ FROM MESSAGES AND LEAVE ONLY PAYLOAD.

%TODO NEEDS A MONITOR FOR IMPOSING THIS CONSTRAINT. BUT MAYBE WE DON'T NEED IT IF WE ARE JUST INTERESTED IN DEFINING A SYSTEM THAT ADMITS BORDERLINE VIOLATIONS (AND ANYTHING IN ADDITION)


%Let us first consider the case of a causal delivery violation. Since $t$ is borderline, it is necessarily of the following form:
%\begin{align*}
%t_1\cdot \send{i_1}{i_1,j,v_1} \cdot t_2 \cdot \send{i_2}{i_2,j,v_2} \cdot t_3\cdot \rec{j}{i_2,j,v_2}
%\end{align*}
%where $\send{i_1}{i_1,j,v_1}\leadsto \send{i_2}{i_2,j,v_2}$ and $t_2\cdot t_3$ doesn't contain $\rec{j}{i_1,j,v_1}$
%(otherwise, we would have $\rec{j}{i_1,j,v_1}\leadsto \rec{j}{i_2,j,v_2}$ because these two actions are made by the same process $j$).


\subsection{Checking for Synchronizability Violation Patterns}

We define a monitor in the form of a register automaton (a finite state machine equipped with a set of registers)
that 