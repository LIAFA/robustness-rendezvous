%!TEX root = main.tex
\section{Checking Synchronizability}


Verifying a (bounded state) message passing system using a monitor that maintains the conflict-graph of a trace 
is undecidable since it requires dealing with known undecidable features, e.g., FIFO queues.
We show in this section that checking synchronizability can be reduced 
to a reachability problem in a system that executes under the synchronous semantics, and 
thus doesn't use unbounded message buffers. This result is based on the following ideas
\begin{itemize}
	\item consider a class of boarderline violations to synchronizability, i.e., traces which
are not synchronizable and every strict prefix is synchronizable.
	\item starting from the original system $A$, define a new system $A_0$ executing under
the synchronous semantics, which simulates minimal synchronizability violations of
$S$, if any ($S_0$ goes to an error state whenever such a violation exists).
\end{itemize}

\subsection{Synchronizability Violation Patterns}


\begin{figure}[t]
Causal delivery violation:
\begin{align*}
\send{i_1}{i_1,j,v_1}\leadsto \send{i_2}{i_2,j,v_2}\mbox{ and }
\rec{j}{i_2,j,v_2} \leadsto \rec{j}{i_1,j,v_1} 
\end{align*}
Exchange pattern:
\begin{align*}
\send{i_1}{i_1,j_1,v_1}\leadsto \rec{j_2}{i_2,j_2,v_2}\mbox{ and }
\send{i_2}{i_2,j_2,v_2}\leadsto \rec{j_1}{i_1,j_1,v_1}
\end{align*}
\caption{Synchronizability violation patterns.}
\label{fig:patterns}
\end{figure}

We show that any trace violating synchronizability contains one of the two violation patterns in Figure~\ref{fig:patterns}.
Intuitively, the first pattern describes a violation to \emph{causal delivery} where two causally related messages sent to the same process
are received in an order different from that in which they were sent. Here, the causal order between messages is defined
by the paths in the action graph. The second violation pattern describes a situation in which two messages are sent concurrently 
and each message is received after the other one is sent. This situation arises for instance in protocols solving consensus which alternate
between phases in which all processes send messages to their peers and phases in which they receive messages from their peers. 
In these cases, the sends and receives in the exchange pattern are causally related since they are executed by the same process.

We say that a trace contains a causal delivery violation or an exchange pattern if it contains the actions and the action-graph paths
specified in Figure~\ref{fig:patterns}.

\begin{theorem}\label{th:patt}
A trace is not synchronizable if{f} it contains a causal delivery violation or an exchange pattern.
\end{theorem}
\begin{proof}
TODO
\end{proof}

\subsection{Simulating Borderline Violations}

A violation to synchronizability $t$ is called \emph{borderline} when every strict prefix of $t$ is synchronizable.
We first show that any borderline synchronizability violation of a message passing system $A$ can be simulated by 
the synchronous semantics of another message passing system $something(A)$ computable in polynomial time.

Let $t$ be a borderline synchronizability violation.
By Theorem~\ref{th:patt}, $t$ contains either a causal delivery violation or an exchange pattern.
Therefore, we have 
\begin{align*}
t=t_1\cdot \send{i_1}{i_1,j_1,v_1} \cdot t_2\cdot \rec{j_1}{i_1,j_1,v_1}
\end{align*} 
where $t'=t_1\cdot \send{i_1}{i_1,j_1,v_1} \cdot t_2$ is a synchronous trace.
The prefix $t'$ can be simulated by the synchronous semantics of a message passing system $A'$ 
where every send transition in $A$ is cloned to an epsilon transition, i.e., if $A=(T,l_0,\<AdtM>)$
then $A'=(T',l_0,\<AdtM>)$ where
\begin{align*}
T'=T\cup\set{(l,\epsilon,l'): \exists s\in S.\ (l,s,l')\in T}.
\end{align*}

To simulate the last receive in $t$, we transform $A'$ such that any message sent to some process $j_1$ can be redirected to some 
distinguished process $i_0$ who relays it to $j_1$ non-deterministically at a later time. We constrain the system such that it can
redirect only one message.

TODO NEEDS A MONITOR FOR IMPOSING THIS CONSTRAINT. BUT MAYBE WE DON'T NEED IT IF WE ARE JUST INTERESTED IN DEFINING A SYSTEM THAT ADMITS BORDERLINE VIOLATIONS (AND ANYTHING IN ADDITION)


%Let us first consider the case of a causal delivery violation. Since $t$ is borderline, it is necessarily of the following form:
%\begin{align*}
%t_1\cdot \send{i_1}{i_1,j,v_1} \cdot t_2 \cdot \send{i_2}{i_2,j,v_2} \cdot t_3\cdot \rec{j}{i_2,j,v_2}
%\end{align*}
%where $\send{i_1}{i_1,j,v_1}\leadsto \send{i_2}{i_2,j,v_2}$ and $t_2\cdot t_3$ doesn't contain $\rec{j}{i_1,j,v_1}$
%(otherwise, we would have $\rec{j}{i_1,j,v_1}\leadsto \rec{j}{i_2,j,v_2}$ because these two actions are made by the same process $j$).


\subsection{Checking for Synchronizability Violation Patterns}

We define a monitor in the form of a register automaton (a finite state machine equipped with a set of registers)
that 