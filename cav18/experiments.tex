%!TEX root = main.tex
\section{Experimental Evaluation}

\begin{wrapfigure}{l}{5.7cm}
\vspace{-8mm}
{\small
\begin{tabular}{|c|c|c|c|c|}
\hline
Name & Loc & Instr & $k$ & Time \\
\hline
Elevator&  \\
\hline
Two-phase commit &  \\
\hline
Replication Storage& \\
\hline
German Protocol &  \\
\hline
OSR &  \\
\hline
\end{tabular}
}
\caption{Experimental results.}
\label{fig:experiments}
\end{wrapfigure}
As a proof of concept, we have applied our procedure for checking $k$-synchronizability to a set of examples extracted from the distribution of the P language~\footnote{Available at \url{https://github.com/p-org}.}. In the absence of an exhaustive model-checker for this language, we have rewritten these examples in the Promela language and used the Spin model checker~\footnote{Available at \url{http://spinroot.com}} for discharging the reachability queries. For a given Promela program, its $k$-synchronous semantics is implemented as an instrumentation which uses additional boolean variables to enforce that sends and receives interleave in $k$-exchange phases. Then, the monitors defined in Section~\ref{sec:verif} are defined as additional processes which observe the sequence of $k$-exchanges in an execution and update their state accordingly. Finding a conflict-graph cycle which witnesses non $k$-synchronizability corresponds to violating an assertion.
