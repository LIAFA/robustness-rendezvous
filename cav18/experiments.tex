%!TEX root = main.tex
\section{Experimental Evaluation}

\begin{wrapfigure}{l}{4.8cm}
\vspace{-7mm}
{\scriptsize
\begin{tabular}{|c|c|c|c|c|c|}
\hline
Name & Proc & Loc  & $k$ & Time \\
\hline
Elevator& 3 & 90  & 2 & 64.3s  \\
\hline
OSR & 4 & 63  & 1 & 1.28s \\
\hline
German & 5 & 335 & 2 & 38m \\
\hline
Two-phase commit &4 & 57 & 1 & 1.43s  \\
\hline
Replication storage & 6 & 100  & 4 & 92.8s \\
\hline
\end{tabular}
}
\caption{Experimental results.}
\label{fig:experiments}
\vspace{-7mm}
\end{wrapfigure}
As a proof of concept, we have applied our procedure for checking $k$-synchronizability to a set of examples extracted from the distribution of the P language~\footnote{Available at \url{https://github.com/p-org}.}. Two-phase commit and Elevator are presented in Section~\ref{sec:motivation}, German is a model of the cache-coherence protocol with the same name, OSR is a model of a device driver, and Replication Storage is a model of a protocol ensuring eventual consistency of a replicated register.
These examples cover common message communication patterns that occur in different domains: distributed systems (Two-phase commit, Replication storage), device drivers (Elevator, OSR), cache-coherence protocols (German). %They are abstractions of practical systems. 
%The benefits of our approach extend to real-size systems whose executions are "compositions" of such communication patterns.
%
We have rewritten these examples in the Promela language and used the Spin model checker~\footnote{Available at \url{http://spinroot.com}} for discharging the reachability queries. For a given  program, its $k$-synchronous semantics and the monitors defined in Section~\ref{sec:verif} are implemented as ghost code. 
%is implemented as an instrumentation which uses additional boolean variables to enforce that sends and receives interleave in $k$-exchange phases. Then, the monitors defined in Section~\ref{sec:verif} are defined as additional processes which observe the sequence of $k$-exchanges in an execution and update their state accordingly. 
Finding a conflict-graph cycle which witnesses non $k$-synchronizability corresponds to violating an assertion. 

The experimental data is listed in Figure~\ref{fig:experiments}: Proc, resp., Loc, is the number of processes, resp., the number of lines of code (loc) of the original program, $k$ is the \emph{minimal} integer for which the program is $k$-synchronizable, and Time gives the number of minutes needed for this check. The ghost code required to check $k$-synchronizability takes 250 lines of code in average.
