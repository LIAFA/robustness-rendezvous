%!TEX root = main.tex
\section{Motivating examples}

%TODO GIVE AN EXAMPLE FOR 1 ROBUSTNESS AND 2 ROBUSTNESS IN THE P LANGUAGE (EXPLAIN A LITTLE BIT THE SYNTAX)
%
%TALK ABOUT THE MOST USUAL WAY OF SEEING THEM
%
%TAKE SOME EXECUTIONS AND SHOW HOW THEY CAN BE REORDERED AND EXECUTED ON A STRONGER SEMANTICS
%
%EXPLAIN THE STRONGER SEMANTICS
%
%SAY THAT FINDING VIOLATIONS MEANS DETECTING SOME PARTICULAR CLASS OF CYCLES
%
%SAY WHAT ARE THE CONSEQUENCES: SAFETY, DEADLOCK
%
%SAY THAT FINDING SUCH CYCLES CAN BE DONE ON THE STRONGER SEMANTICS - GIVE THE MAIN IDEAS

We provide in this section examples illustrating the relevance and the applicability of our approach. %These examples correspond to several types of protocols used in practice. 
Figure \ref{fig:commit} shows a {\em commit protocol} allowing a client to update a memory that is replicated in two nodes. The access to these nodes is controlled by a manager. Figure \ref{fig:commit-exec} shows an execution of this protocol. This system is 1-synchronizable, i.e., every execution of this system is equivalent to one where only rendezvous communication is used. Intuitively, this holds because mutually interacting components are never in the situation where messages sent from one side to the other one are crossing messages sent in the other direction (i.e., the components are "talking" to each other at the same time). For instance, the execution in \ref{fig:commit-exec} is 1-synchronizable because its \emph{conflict graph} (shown in the same figure) is acyclic. Nodes in the conflict graph are matching send-receive pairs (numbered from 1 to 6 in the figure), and edges correspond to the program order between actions in these pairs. The conflict graph being acyclic means that matching pairs of send-receive actions are ``serializable'', which implies that it is equivalent to an execution where every send is immediately followed by the matching receive (as in rendezvous communication).

\begin{figure}[t]
\begin{center}
\includegraphics[width=8.5cm]{commit.pdf}
\end{center}
\vspace{-5.5mm}
\caption{A distributed commit protocol. Each process is defined as a labeled transition system. Transitions are labeled by send and receive actions, e.g., $\senda{\sf{c},\sf{m},\sf{update}}$ is a send from the client $\sf{c}$ to the manager $\sf{m}$ with payload $\sf{update}$. Similarly, $\reca{c,\sf{OK}}$ denotes process $\sf{c}$ receiving a message 
$\sf{OK}$.}
\label{fig:commit}
\vspace{-3.5mm}
\end{figure}

\begin{figure}[t]
\begin{center}
\includegraphics[width=7cm]{MSC-commit.pdf}
\end{center}
\vspace{-5mm}
\caption{An execution of the distributed commit protocol and its conflict graph.}
\label{fig:commit-exec}
\vspace{-7mm}
\end{figure}

Although the message buffers are bounded in all the computations of the commit protocol, this is not true for every 1-synchronizable system.
%Now, it can be observed that in the commit protocol message buffers are bounded in all the possible computations. This is in general not the case for 1-synchronizable systems. It might be the case indeed that 
There are asynchronous computations where buffers have an arbitrarily big size, which are equivalent to synchronous computations. This is illustrated for instance by a (family of) computations shown in Figure~\ref{fig:elevator-exec1} of the elevator system shown in Figure \ref{fig:elevator} (a simplified version of the system described in~\cite{DBLP:conf/pldi/DesaiGJQRZ13}).  In this execution, the user keeps sending requests for closing the door, which generates an unbounded sequence of messages in the entry buffer of the elevator process. However, these computations are synchronizable since they are equivalent to a synchronous computation where the elevator receives immediately every message sent by the user. This is witnessed by the acyclicity of the conflict graph of this computation (shown on the right of the same figure). It can be checked that the elevator system shown in Figure \ref{fig:elevator} is a 1-synchronous system (without the dashed edge). 

Consider now a slightly different version of the elevator system where the transition from {\sf Stopping2} to {\sf Opening2} is moved to target {\sf Opening1} instead of {\sf Opening2} (see the dashed transition in Figure \ref{fig:elevator}). It can be seen that this version has the same state space as the previous one. Indeed, moving that transition from {\sf Stopping2} to {\sf Opening1} gives the possibility to {\sf Elevator} to send a message open to {\sf Door}, but the latter can only be between {\sf StopDoor} and {\sf ResetDoor} at this point, and therefore it can (maybe after sending {\sf doorStoped} and {\sf doorOpened}) receive at state {\sf ResetDoor} the message {\sf open} and stay in the same state. However, this version of the system is not 1-synchronizable as it is shown in Figure \ref{fig:elevator-exec2}: Suppose that {\sf Door} is at state {\sf StopDoor}, and that {\sf Elevator} is at state {\sf Stopping2}. Then, {\sf Door} can send a message {\sf doorStoped} and move to the state {\sf OpenDoor}. Next, {\sf Elevator} can receive that message and move to state {\sf Opening1}. At this point, {\sf Elevator} and {\sf Door} can only exchange messages: message {\sf doorOpened} from {\sf Door} to {\sf Elevator} and message {\sf open} from {\sf Elevator} to {\sf Door}. The conflict graph of this execution, shown on the right of Figure \ref{fig:elevator-exec2}, contains a cycle of size 2 between the two matching pairs of send-receive actions involved in the exchange interaction. 

\begin{figure}[t]
\includegraphics[width=12cm]{elevator.pdf}
\vspace{-3mm}
\caption{The Elevator example}
\label{fig:elevator}
\vspace{-5mm}
\end{figure}

\begin{figure}[t]
\begin{subfigure}[t]{6cm}
\includegraphics[width=6cm]{MSC-elevator1.pdf}
\caption{A $1$-synchronizable execution.}
\label{fig:elevator-exec1}
\end{subfigure}
\hspace{1cm}
\begin{subfigure}[t]{5cm}
\includegraphics[width=5cm]{MSC-elevator2.pdf}
\caption{A computation with a 2-exchange.}
\label{fig:elevator-exec2}
\end{subfigure}
\caption{Executions of the Elevator.}
\label{fig:elevator-exec}
\vspace{-5mm}
\end{figure}

%\begin{figure}
%\includegraphics[width=10cm]{replication.pdf}
%\caption{A replication storage protocol}
%\label{fig:replication}
%\end{figure}
%
%\begin{figure}
%\includegraphics[width=7cm]{MSC-storage.pdf}
%\caption{An execution of the replication storage protocol and its conflict graph.}
%\label{fig:replic-exec}
%\end{figure}



%We have investigated other examples defined in the P language~\footnote{Available at \url{https://github.com/p-org}.}, e.g., a replication storage protocol (see Appendix~\ref{asec:motivation}), an implementation of the German cache coherence protocol and an implementation of a device driver, and we have established manually and through testing that they are $k$-synchronizable~\footnote{For the testing part, we have checked on tens of thousands of executions generated through randomized testing, each execution with thousands of steps, that the corresponding conflict-graph satisfies the properties required for $k$-synchronizability.}. The value of $k$ is most of the times $1$ or $2$.

%\cite{PRuntime}

%\begin{figure}
%\includegraphics[width=10cm]{commit.pdf}
%\caption{A distributed commit protocol.}
%\label{fig:commit}
%\end{figure}
%
%\begin{figure}
%\includegraphics[width=7cm]{MSC-commit.pdf}
%\caption{An execution of the distributed commit protocol and its conflict graph.}
%\label{fig:commit-exec}
%\end{figure}

%\begin{figure}
%\includegraphics[width=13cm]{elevator.pdf}
%\caption{The Elevator example}
%\label{fig:elevator}
%\end{figure}
%
%\begin{figure}
%\begin{subfigure}[t]{6cm}
%\includegraphics[width=6cm]{MSC-elevator1.pdf}
%\caption{A synchronizable execution.}
%\label{fig:elevator-exec1}
%\end{subfigure}
%\hspace{1cm}
%\begin{subfigure}[t]{5cm}
%\includegraphics[width=5cm]{MSC-elevator2.pdf}
%\caption{A computation with a 2-exchange.}
%\label{fig:elevator-exec2}
%\end{subfigure}
%\caption{Executions of the elevator.}
%\label{fig:elevator-exec}
%\end{figure}

